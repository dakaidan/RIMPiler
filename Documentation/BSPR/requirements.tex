\chapter*{Requirements}

This project aims to produce a compiler for the reversible programming language RIMP in a modular way, allowing for the use of different backends, and frontends, and to easily alter various aspects of the compilers pipeline.

\section*{Functional Requirements}

\begin{enumerate}
    \item The compiler must be able to produce a token stream from a RIMP source file.
    \item The compiler must be able to produce an abstract syntax tree from a token stream.
    \item The compiler must be able to perform semantic transformations on the abstract syntax tree to produce a reversible program.
    \item The compiler must be able to produce the reverse abstract syntax tree from an abstract syntax tree.
    \item The compiler must be able to produce an SSA form from an abstract syntax tree.
    \item The compiler must be able to produce an executable file from an input file.
    \item The compiler should be able to target a reversible SSA form such as RSSA.
    \item The compiler should provide a backend for a reversible SSA form such as RSSA to produce a reversble machine code such as PISA.
\end{enumerate}


\section*{Non-Functional Requirements}

\begin{enumerate}
    \item The compiler should be able to compile in a reasonable amount of time.
    \item The compiler should be able to give feedback to the user when compilation fails.
\end{enumerate}

%An overview of functional requirements, perhaps non-functional requirements, and a brief description of use cases.
%
%Note, From a example submission:
%Ullah Joya
%Loyalty Card App
%2015
%
%"Importance of the requirements is shown next to them. A high priority requirement
%must be implemented for the project to be considered complete. Without low
%priority requirements, useful but not critical, the application should offer basic
%23
%functionality. High priority requirements should be implemented first and then
%tackle other lower priority requirements."
%
%So we should be able to specify even aspirational requirements, and then prioritise them.