\chapter*{Background and Context}

\section{Reversible Computing}\label{sec:reversible-computing}
Reversible computing is a model of computation in which the original state of the system can be recovered from the final state of the system\cite{bennett_logical_1973}.
Reversible computing is a wide-reaching, yet relatively under researched area of computer science with the potential to impact many areas of computer science, including low-power computing\cite{landauer_irreversibility_1961}, high-speed computing, quantum computing, and more.
% Cite some papers here for each of these maybe by Kaylan S. Perumalla, he seems to be around the classical aspects

\subsection{Classical Computing}\label{subsec:classical-computing}
In classical computing reversible computing has many applications, but one of the most prominent, which cannot be overcome by other means, is to reduce power consumption below the Landauer limit\cite{landauer_irreversibility_1961}.
The ability to potentially reduce power consumption below the Landauer limit has many applications, including both low-power computing, which in the age of mobile computing is a very important area, and high-speed computing, which has applications in all performance critical areas of computing.

% cite this article https://spectrum.ieee.org/landauer-limit-demonstrated
While we are still a ways from achieving this, as the landauer limit is around 100,000 time less energy per bit erasure than current computers.
The study of all facets of reversible computing is key to achieving this goal.

\subsection{Quantum Computing}\label{subsec:quantum-computing}
% Discuss the benefit and importance of quantum computing
Quantum computing allows us to solve problems which are intractable on classical computers, such as integer factorisation, which is the basis of many modern encryption algorithms.
The quantum model of computing largely relies on the reversible model of computation in order to prevent decoherence.

Reversible computing allows the system to be run deterministically in either direction, ensuring there is no loss of information.
% cite Nielsen M.A., Chuang I.L. Quantum Computation and Quantum Information. Cambridge University Press; Cambridge, UK: 2000. here
In quantum computing, the conservation of information is vital to maintaining coherence of the system, and thus, reversible computation plays a key role in quantum computing.

\section{Programing Languages}\label{sec:programing-languages}
Programming languages are a key part of how we build systems, they abstract away much of the complexity or messiness of the underlying system, and provides us with a language which we can quickly build intuition and reason about.
We can see in languages such as Janus\cite{lutz_janus_1986} the language offers a high level inferable syntax which allows us to reason about a program.
However, what it does not provide is a complete abstraction from the model of computation, this can be seen through restrictions in things like assignment.
While this does not restrict the ability of the language, it does add additional barriers to developers, requiring them to adjust to the model of computation being used.

\section{Compilers}\label{sec:compilers}
Compilers are vital to working with high-level programming languages, applying translations to the code to allow it to be run on the target system which would be impractical to do manually for any non-trivial program.
Additionally, compilers can do much more such as optimise the code, and provide feedback to the developer such as warnings and errors.
This allows the developer to produce better code with less issues than it may have otherwise had.

% Give background information, and context of the motivation for the project.

% Background: Information and definitions almost for any relevant domain specific information. Motivation: Why is this important, what is missing, what is new, and why do I want to do this?
