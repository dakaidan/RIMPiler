\chapter{Introduction}
% This is one of the most important components of the report. It should begin with a clear statement of what the project is about so that the nature and scope of the project can be understood by a lay reader. It should summarise everything that you set out to achieve, provide a clear summary of the project's background and relevance to other work, and give pointers to the remaining sections of the report, which will contain the bulk of the technical material.

In this thesis, we aim to provide a framework for the development of reversible languages, focussing on the imperative language \rimplang. The overarching objective of this project is to establish a comprehensive system for developers using \rimplang, as well as for future developers of \rimplang, facilitating seamless modification, and expansion to the language. 

The system is envisioned as a cohesive framework of libraries and command-line interface (CLI) which utilises these libraries. This provides developers with tools for navigating the intricacies of development, from conception, to debugging, to release.

We build on the previous work done by Fern{\'a}ndez et al.\cite{Rimp}, by defining a new syntax for the language, similar to that of familiar languages like C. Additionally, we extend the language to support multiple types, including integers and floating-point numbers, expanding the expressiveness of the language. In order to enable the use of multiple types, we have had to define a formal type system for \rimplang, and alter its structural semantics and abstract machine to facilitate this. We also provide an implementation for this in Rust.

Central to this thesis are three components: an evaluator, an abstract machine, and a compiler targeting the Java Virtual Machine (JVM). Each part is important for the development of \rimplang programs, contributing to a cohesive environment.

The evaluator serves as a fast way to validate the results of your program, allowing developers to iterate and explore their programs, and the \rimplang language.
Complementary to this, the abstract machine allows for bidirectional debugging of \rimplang programs. By utilising the reversible nature of \rimplang, developers are able to step forwards and backwards through their programs and introspect the program state in order to debug, and understand the computation. Finally, the compiler acts as the final step in this process, allowing developers to compile their programs to a common and portable target to be run on many systems while still being reversible.

Through this thesis, we are faced with many challenges. Foremost, is the difficulties of extending the theoretical foundations provided in the seminal \rimplang paper\cite{Rimp} to our needs, while also addressing, and overcoming the nuances of a practical implementation.
