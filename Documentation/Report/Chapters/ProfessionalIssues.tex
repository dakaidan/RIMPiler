\chapter{Legal, Social, Ethical \& Professional Issues}
% Your report should include a chapter with a reasoned discussion about legal, social ethical and professional issues within the context of your project problem. You should also demonstrate that you are aware of the regulations governing your project area and the Code of Conduct \& Code of Good Practice issued by the British Computer Society, and that you have applied their principles, where appropriate, as you carried out your project.

%2 pages

\section{BCS Code of Conduct}

Throughout this project, I aim to ensure all relevant standards of the BSC Code of conduct are upheld, and as such below is an overview of how I have followed the standards.

\begin{enumerate}
    \item Public Interest
    \begin{enumerate}
        \item have due regard for public health, privacy, security and wellbeing of others and the environment: \textit{N/A.}
        \item have due regard for the legitimate rights of Third Parties: \textit{Throughout the course of the project, I have ensured to follow all copyright notices and licences on software used.}
        \item conduct your professional activities without discrimination on the grounds of sex, sexual orientation, marital status, nationality, colour, race, ethnic origin, religion, age or disability, or of any other condition or requirement: \textit{N/A.}
        \item promote equal access to the benefits of IT and seek to promote the inclusion of all sectors in society wherever opportunities arise: \textit{Not directly applicable to the development process, however, section \ref{LPC} and \ref{Accessibility} have further details.}
  \end{enumerate}
  \newpage
  \item Professional Competence and Integrity
  \begin{enumerate}
      \item only undertake to do work or provide a service that is within your professional competence: \textit{I have ensured to extensively research existing works to gain relevant competencies.}
      \item NOT claim any level of competence that you do not possess: \textit{N/A.}
      \item develop your professional knowledge, skills and competence on a continuing basis, maintaining awareness of technological developments, procedures, and standards that are relevant to your field: \textit{Carried out an extensive literature review covering all areas of research.}
      \item ensure that you have the knowledge and understanding of Legislation and that you comply with such Legislation, in carrying out your professional responsibilities: \textit{N/A no legislation outside copyright protection is in question with this project.}
      \item respect and value alternative viewpoints and, seek, accept and offer honest criticisms of work: \textit{N/A.}
      \item avoid injuring others, their property, reputation, or employment by false or malicious or negligent action or inaction: \textit{N/A.}
      \item reject and will not make any offer of bribery or unethical inducement: \textit{N/A}
  \end{enumerate}
  \item Duty to Relevant Authority
  \begin{enumerate}
      \item carry out your professional responsibilities with due care and diligence in accordance with the Relevant Authority’s requirements whilst exercising your professional judgement at all times: \textit{N/A.}
      \item seek to avoid any situation that may give rise to a conflict of interest between you and your Relevant Authority: \textit{N/A.}
      \item accept professional responsibility for your work and for the work of colleagues who are defined in a given context as working under your supervision: \textit{N/A.}
      \item NOT disclose or authorise to be disclosed, or use for personal gain, or to benefit a third party, confidential information except with the permission of your Relevant Authority, or as required by Legislation: \textit{N/A.}
      \item NOT misrepresent or withhold information on the performance of products, systems or services (unless lawfully bound by a duty of confidentiality not to disclose such information), or take advantage of the lack of relevant knowledge or inexperience of others: \textit{N/A.}
  \end{enumerate}
  \item Duty to the Profession
  \begin{enumerate}
      \item accept your personal duty to uphold the reputation of the profession and not take any action which could bring the profession into disrepute: \textit{N/A.}
      \item seek to improve professional standards through participation in their development, use and enforcement: \textit{evaluated and referenced the BSC code of conduct before and after the project to ensure compliance.}
      \item uphold the reputation and good standing of BCS, the Chartered Institute for IT: \textit{N/A.}
      \item act with integrity and respect in your professional relationships with all members of BCS and with members of other professions with whom you work in a professional capacity: \textit{N/A.}
      \item encourage and support fellow members in their professional development: \textit{N/A.}
  \end{enumerate}
\end{enumerate}

\section{Licensing}

\rimp is available in its entirety under a permissive MIT licence, and all source, and documentation will be made available as soon as possible in furtherance of research and learning in the area of reversible computing.

\section{Accessibility}\label{Accessibility}

\rimp aims to be an evolving system in which may contributors could work together to improve the framework. To this aim, \rimp aims to follow where possible the concepts of literate programming\cite{LiterateProgramming}. This enables the software and techniques to be more accessible to more users, allowing them to contribute, adapt, and learn from the code base.

\section{Low Power Computing}\label{LPC}

Reversible computing, while not fully present, promises to enable low-power computing through the reduction of heat loss\cite{landauerIrreversibility}. As such, the progress and research within this area has the potential to revolutionise many aspects of computing. 
Two primary considerations are the sustainability of computing and its accessibility to economically disadvantaged communities.
Low-power computing could allow for a significant reduction in C0$_2$ emissions\cite{EnvironmentalComputing} primarily by targetting the energy used during the lifetime of the device.
Further, due to the reduction in energy consumption, we enable computer access to communities which may not have previously had as much access due to high energy costs. The reduction in costs could help allow access to computing devices to those from disadvantaged backgrounds\cite{USDigitalInequality, USRacialDigitalInequality} to access services, education, and financial opportunities\cite{DigitalInequalitySouth}. 
Energy also limits access to computing applications, for example, fuzzing\cite{GreenFuzzer} and AI\cite{sustainableAI}, reducing the energy consumption of such tasks would significantly reduce the barrier to entry, enabling people to use this technology in a wider range of areas than previously possible.
As such, \rimp is a small step in the direction of providing more equal computing.